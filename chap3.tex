\chapter{Conclusion and Outlook}
\label{chap:conclusion}

\section{Technical Thoughts}
\begin{flushleft}
	The system can easily be set up to be much more fault-tolerant by increasing the amount of replicas of each Service in the system.
	Most replica values used throughout this are only values of 1, but there are quite a few services that comprise the system. So for the
	benefit of the machine on which the testing was carried out the values were set to 1.
	\bigbreak
	More system metrics would be the logical next step in the system. The use of Prometheus and Grafana would be a good way to monitor and
	find ways to improve performance.
	\bigbreak
	The Makefile should be extended and added to. The aim should be to try and create the entire system in a single command (once Kubernetes
	is running). This does seem feasible, however, plenty of time would be taken testing how long different components take to start up and if
	retires/ back-offs are required as a result.
	\bigbreak
	A consumer with some basic functionality in processing the stream data would be a great addition here.
	\bigbreak
	As to would the addition of TLS between each component. Maybe this could be with the incorporation of a Service Mesh into the architecture.
	This was actually looked into and one of the reasons for not using Strimzi from the start was its seemingly incompatibility with Istio,
	the service mesh with which I have the most experience. If Istio was to be incorporated that would enable the easy automation of Prometheus,
	Grafana and a whole host of other metric gathering tools which come natively with an Istio deployment.
	\section{Personal Thoughts}
	I could have written so much more. I knew that the system is a big one to implement and given less time and word constraints I would
	have liked to explore this area much more.\newline
	I am actually quite proud of the system that was created. These components(scaled up versions) are used in industry for the deployment
	of some of the largest and highly trafficked systems in the world. It was a sensational learning experience putting it all together.
	\bigbreak
	The amount of knowledge I have gained about Kubernetes and microservice architectures is much, much more than I thought I would learn at the
	outset of this report, and it will surely stand by me as I, hopefully, make my way into industry.\newline
	My current understanding around the operator pattern is making me believe it is a game changer. The ability to automate/ abstract layers
	of components in the system is a very powerful one. It is one I only began to realise when I noticed that the Strimzi operator deploys
	and manages other operators (Kafka Entity Operator). The ease at which the system can be scaled up and down is frighteningly straightforward.
	The learning curve is a little steep, but once one is familiar enough with the Kubernetes basics the architecture becomes a thing of beauty.
	\bigbreak
	I am also quite happy that every goal as set out in the introduction was achieved and that all components used are open source.
	A fascinating insight in to the open source world came when I was rather disillusioned about the Kafka Connect configurations. I reached out to the
	Debezium community and was answered to by none other than the lead-maintainer and co-founder of the Debezium project, Gunnar Morling. I asked
	what I thought was maybe a stupid simple question, but it turns out it was an issue that had caught others before. Gunnar pointed me to
	the Debezium community group chat where I found an article which helped me overcome the issue.\newline
	The knowledge that one can build a system that is capable of slotting into the back end of Fortune 100 companies, from completely free
	and open sourced software is truly amazing. The whole area of systems architecture was one I was intrigued about before this report. But now
	I'm even more intrigued now.
\end{flushleft}