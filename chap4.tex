
\chapter{Conclusion}
\label{chap:conclusion}
This section sees some reflection both technical and personal through a personal perspective.
\section{Technical}

\begin{quoting}
How does the system match expectations?
\end{quoting}
The system performs to a very decent standard. Approximately 85\% of invoices are correctly labeled. Considering the amount of labels 
that are in use and that this is the very first version of the model, I am happy with the result.\\
Another factor to keep in mind is the low amount of data in the training set. With a better dataset --- both more samples 
and more solid annotation\footnote{The annotating is something that I had never really thought about previous to this project.
It is not the most straightforward of tasks when one is unfamiliar with how the model learns. I have other ideas on how to change 
the annotation standard and compare performance metrics.}, the model's performance should increase. \\
Furthermore, an interesting observation that will cut down the time to retrain a model is the fact that I can now use 
the model to predict all labels on a given invoice. These labels are exactly the same as used in annotation,
therefore, I can now use the model to predict the labels on all new and non-annotated invoices. I can 
alter the annotation slightly (if needed) and build up a larger dataset much quicker by automating the annotation process.\\
The more data (up to a point) should lead to more accurate results, resulting in an even easier task of correcting annotations 
for the next rounds of training.
\bigbreak
It is important to note that we have not completely automated these processes. But we have replaced a very slow process 
with a good level of automation and a decrease in complexity (checking and correcting rather than full extraction).
\begin{quoting}
How well did the chosen technologies fit together to realise the desired architecture?
\end{quoting}
The technologies fit together very well. Following a microservice architecture throughout the project, it becomes 
quite easy to upgrade components and even change the architecture whilst effecting minimal components directly.\\
A slight negative was the workaround which had to be used due to the Minikube issue, but with the 
troubleshooting, invariably, comes a greater understanding of the components which constitute the issue. 
Kubernetes handled this upgrade very smoothly. 
\\
A case in point that has not been mentioned, a new version of the Transactions API\footnote{Entry point into the 
global system for transactional data from PoS systems.} was rolled out. This was actually quite a substantial change to 
the data types and the structure of the data in the system. The new API was upgraded in seconds, migrating 
and introducing changes to the database schema with no errors or issues.


\section{Personal}
On a personal note, this project has hit the objectives that I have set for it, and then some. Having a keen interest in the area 
of machine learning, gaining more knowledge about the area was one of the main goals of this project. From having 
no experience in the field a year ago, I have managed to implement an entire pipeline that may actually serve as a 
prototype to a real-world product.\\ 
I have also familiarised myself with the technologies used in industry. I have also gained a massive amount of 
understanding in the area of data manipulation and improved my Python skills considerably. 
\bigbreak
Elements of the project (the ML pipeline) have already been used in real life scenarios in getting documents ready for VAT returns. Working 
almost flawlessly\footnote{It may be noted that the 85\% \FO was calculated against the models ability to predict every label.
For the VAT returns, only 5 labels were needed, and the model actually did a very good job for these labels ---anecdotally.}, going forward I look 
forward to building on the implementation of the current solution. This will make my life easier and gives me back time 
that I otherwise would have had to spend on these manually completing these processes.
\label{sec:personal}
\bigbreak
I remember fearing that I wouldn't be able to grasp the concepts that underpin machine learning, model architecture and 
how models work. With the transformers architecture family being regarded as the current state-of-the-art, 
not only have I gained a great deal of knowledge in the field of machine learning, but I now have a very good understanding 
of the architectures of models that are at the cutting edge of machine learning.\\
Being able to have conversations with Walid, about scoring, training and evaluation methods was a big confidence 
boost. There is, obviously, 
a massive amount still to learn, but I look forward to doing so with a newly obtained confidence in my abilities. 
